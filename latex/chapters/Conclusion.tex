% !Mode:: "TeX:UTF-8"


\chapter{结语}
这一章作为本文的最后一章,首先总结了这篇文章的主要工作,然后以上述工作为基础,讨论了接下来即将开始的工作。

\section{本文工作总结}

这篇文章以聚类分析为中心,以轨迹数据 (行为数据) 为应用对象,尝试提出了新的聚类算法以及相似度衡量方法,并将这两者用于轨迹数据,尝试挖掘轨迹数据中存在的规律。具体来说,本文第一部分 (第 \ref{chp:Similarity}章) 介绍了轨迹数据的相似度度量方法,第二部分 (第 \ref{chp:Attractor}章) 介绍了一个新的聚类算法,之后 (第 \ref{chp:Experiments}章) 将这两者应用于轨迹数据,尝试发现轨迹中存在的规律。

\vspace{2mm}
\begin{itemize}
\vspace{1mm}
\item \hspace{2mm} 第一部分 (第\ref{chp:Similarity}章) 介绍了一个新的衡量轨迹相似度的方法。它通过比较轨迹的时间特征和空间特征来确定轨迹之间的相似度。具体来说,它首先将每条轨迹转化为一个行为网络,通过赋予行为网络一些属性使之刻画轨迹的时间特征和空间特征,在此基础上,通过衡量各个轨迹对应的行为网络之间的相似度来确定轨迹之间的相似度。
\vspace{1mm}
\item \hspace{2mm} 第二部分 (第\ref{chp:Attractor}章) 介绍了一个新的图聚类方法: \emph{Attractor}. 它的核心思想是将网络看作一个动态系统,通过交互过程边的距离的变化来发现社团。通过三种交互模式, \emph{Attractor} 可以有效地发现网络存在的社团结构。与经典的算法在大量数据集上的实验也说明了 \emph{Attractor} 算法的优势。
\vspace{1mm}
\item \hspace{2mm} 第三部分 (第\ref{chp:Experiments}章) 是出租车轨迹数据聚类的结果。首先使用 \ref{chp:Similarity}章中的相似度度量方法计算出了轨迹之间的相似度,之后通过给定的阈值,将这些轨迹转化为一个网络,网络中的每个节点对应一条轨迹,之后应用 \ref{chp:Attractor} 章中介绍的图聚类算法 \emph{Attractor} 对这些轨迹对应的网络进行了聚类。
\end{itemize}

\section{下一步工作}
这篇文章中提出的关于轨迹相似度的算法以及图聚类算法都是最初步的工作,以这两个工作为基础,还有许多有意思的内容可以深入挖掘。下面,我们就来介绍下接下来即将进行的工作。\par
\vspace{2mm}
\textbf{对于轨迹相似度度量方法:} 目前我们用行为网络的节点的度分布来表征轨迹的空间特征,用节点的时间分布来反映轨迹的时间特征,为了更好的把握轨迹的特征,接下来我们希望考虑更多的信息,例如轨迹的局部转移特征 (对应于行为网络中边的信息). \par
\vspace{2mm}
\textbf{社团挖掘算法 (\emph{Attractor}):} 这个社团挖掘算法仅仅是一个开端,在现有算法的基础上,还有许多问题可以挖掘,例如从理论上证明算法的收敛性 (算法的理论基础),或是通过并行运算缩短算法的运行时间,或是当网络中存在非常大的度的节点时怎样加速处理等等。接下来,我们希望能在这个算法的理论基础方面得到一些进展。
