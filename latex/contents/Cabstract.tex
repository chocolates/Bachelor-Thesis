% !Mode:: "TeX:UTF-8"

\begin{Cabstract}{数据挖掘}{聚类}{社团挖掘}{相似度度量}{轨迹}
科技的进步为我们创造并积累了大量的数据,因此我们需要更多的有效的方法来挖掘不同形式的数据中的有用的信息,这也是数据挖掘的核心任务。针对日益增加的轨迹数据,这篇文章的第一部分首先提出了一个衡量轨迹数据的相似度方法,它的基本想法是为每条轨迹构建一个行为网络 (\emph{Mobility Network}),用行为网络的特征来反映轨迹的时间和空间信息。通过比较轨迹对应的行为网络的特征,例如节点出现次数的分布、访问节点的时间的分布,进而我们可以得到轨迹的相似度。在第二部分中,这篇文章提出了一个新的图聚类算法 (\emph{Attractor}),其基本思想是将一个网络看作一个动态系统,通过节点与节点之间的交互,处于同一个社团内的节点之间距离将会减小,处于不同社团的节点之间距离将会变大,通过检测节点与节点之间的距离,最终我们可以得到这个网络的社团结构。大量的实验说明了这个算法的优势。在上述两个算法的基础上,我们以一个出租车数据集为例,完成了对出租车轨迹的聚类分析。
\end{Cabstract}




% \begin{Cabstract}{$M-$矩阵}{$H-$矩阵}{Drazin逆}{Pseudo-Drazin逆}{条件数}
% J.\nbs H.\nbs Wilkinson\citeup{iflai1977}建立了非奇异矩阵的逆是矩阵元素的连续函数的理论。G.\nbs W.\nbs Stewart\citeup{crawfprd1995}推出了矩阵 的广义逆 的连续性。为了得到Drazin逆的连续性, 本文先给出了$M-$矩阵、$H-$矩阵类的逆的连续性。Campbell和Meyer\citeup{zhaoyaodong1998} 也给出了Drazin 逆的连续性性质,但没有给出明显的边界。\par
% Drazin逆对扰动是很不稳定的。然而,在某种特定的扰动条件下,矩阵$(A+E)^D$与$A^D$的接近程度能够得到量化且也能得到明显的相对误差边界。基于Drazin逆的不同形式,很多科学家和学者从事这一方面的研究。U.\nbs G.\nbs Rothblum 给出的Drazin逆的以下的表达式:
% $$A^D=(A-H)^{-1}(I-H)=(I-H)(A-H)^{-1}$$
% 其中$H=I-AA^D=I-A^DA$.基于这个表达式,我们在本文中也给出了$\|(A+E)^D-A^D\|_2/\| A^D\|$和$\|(A+E)^\sharp-A^D\|_2/\|A^D\|_2$的范数估计,并与前人的成果进行了比较。
% \end{Cabstract}
