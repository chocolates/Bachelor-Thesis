% !Mode:: "TeX:UTF-8"

\begin{Eabstract}{data mining}{clustering}{community detection}{similarity}{trajectory}
With the prevalence of technologies in our lives, much more data are generated and stored. So we need to develop more effective and efficient methods to acquire the knowledge behind the data. And this is the core task of data mining. The first part of this paper proposes a similarity measure for trajectory data. Its basic idea is to construct a \emph{Mobility Network} for each trajectory, in which nodes represent the locations appeared in the trajectory and edges record the move from one place to another place. We can measure the similarity of two trajectories by comparing their networks like the weight distributions of nodes. In conjunction with the clustering method proposed in the next part, we make an analysis on a taxi data set.  In the second part of this paper, we develop a new graph clustering algorithm (\emph{Attractor}), which works on metric space. The basic idea is to envision the target network as a dynamical system, where each node interacts with its neighbors. The interaction will change the distances among nodes, while the distances will affect the interactions. We will see that nodes sharing the same community move together, thus revealing the community structure automatically.  Extensive experiments demonstrate that our algorithm allows the effective and efficient community detection and has good performance compared to state-of-the-art algorithms. 
\end{Eabstract}



% \begin{Eabstract}{$M-$matrices}{$H-$matrices}{Drazin inverse}{Pseudo-Drazin inverse}{Condition number}
% The theory that the inverse of a nonsingular matrix is continuous function of the elements of the matrix was established by J.\nbs H.\nbs Wilkinson\citeup{iflai1977}. The continuity of the generalized inverse $A^+$ of a matrix $A$ was introduced by G.\nbs W.\nbs Stewart\citeup{crawfprd1995}. In this paper, at first, the continuity of the special matrices inverse, such that $M-$matrices and $H-$matrices, respectively, are provided. Campbell and Meyer\citeup{zhaoyaodong1998} also established the continuity properties of Drazin inverse, but the explicit bound was not given.\par
% The Drazin inverse is unstable with respect to perturbation. However, under some specific perturbation , the closeness of the matrices $(A+E)^D$ and $A^D$ can be proved and the explicit bound the relation error can also be obtained. Based on the different representations of Drazin inverse, many scientists and scholars have worked it research. U.\nbs G.\nbs Rothblum gave the following representation of Drazin inverse:
% $$A^D=(A-H)^{-1}(I-H)=(I-H)(A-H)^{-1}$$
% where $H=I-AA^D=I-A^DA$. Based on the representation, we also obtain the norm estimate of $\|(A+E)^D-A^D\|_2/\| A^D\|$ and $\|(A+E)^\sharp-A^D\|_2/\|A^D\|_2$ and compare with the precedent results.
% \end{Eabstract}
